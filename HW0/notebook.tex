
% Default to the notebook output style

    


% Inherit from the specified cell style.




    
\documentclass[11pt]{article}

    
    
    \usepackage[T1]{fontenc}
    % Nicer default font (+ math font) than Computer Modern for most use cases
    \usepackage{mathpazo}

    % Basic figure setup, for now with no caption control since it's done
    % automatically by Pandoc (which extracts ![](path) syntax from Markdown).
    \usepackage{graphicx}
    % We will generate all images so they have a width \maxwidth. This means
    % that they will get their normal width if they fit onto the page, but
    % are scaled down if they would overflow the margins.
    \makeatletter
    \def\maxwidth{\ifdim\Gin@nat@width>\linewidth\linewidth
    \else\Gin@nat@width\fi}
    \makeatother
    \let\Oldincludegraphics\includegraphics
    % Set max figure width to be 80% of text width, for now hardcoded.
    \renewcommand{\includegraphics}[1]{\Oldincludegraphics[width=.8\maxwidth]{#1}}
    % Ensure that by default, figures have no caption (until we provide a
    % proper Figure object with a Caption API and a way to capture that
    % in the conversion process - todo).
    \usepackage{caption}
    \DeclareCaptionLabelFormat{nolabel}{}
    \captionsetup{labelformat=nolabel}

    \usepackage{adjustbox} % Used to constrain images to a maximum size 
    \usepackage{xcolor} % Allow colors to be defined
    \usepackage{enumerate} % Needed for markdown enumerations to work
    \usepackage{geometry} % Used to adjust the document margins
    \usepackage{amsmath} % Equations
    \usepackage{amssymb} % Equations
    \usepackage{textcomp} % defines textquotesingle
    % Hack from http://tex.stackexchange.com/a/47451/13684:
    \AtBeginDocument{%
        \def\PYZsq{\textquotesingle}% Upright quotes in Pygmentized code
    }
    \usepackage{upquote} % Upright quotes for verbatim code
    \usepackage{eurosym} % defines \euro
    \usepackage[mathletters]{ucs} % Extended unicode (utf-8) support
    \usepackage[utf8x]{inputenc} % Allow utf-8 characters in the tex document
    \usepackage{fancyvrb} % verbatim replacement that allows latex
    \usepackage{grffile} % extends the file name processing of package graphics 
                         % to support a larger range 
    % The hyperref package gives us a pdf with properly built
    % internal navigation ('pdf bookmarks' for the table of contents,
    % internal cross-reference links, web links for URLs, etc.)
    \usepackage{hyperref}
    \usepackage{longtable} % longtable support required by pandoc >1.10
    \usepackage{booktabs}  % table support for pandoc > 1.12.2
    \usepackage[inline]{enumitem} % IRkernel/repr support (it uses the enumerate* environment)
    \usepackage[normalem]{ulem} % ulem is needed to support strikethroughs (\sout)
                                % normalem makes italics be italics, not underlines
    

    
    
    % Colors for the hyperref package
    \definecolor{urlcolor}{rgb}{0,.145,.698}
    \definecolor{linkcolor}{rgb}{.71,0.21,0.01}
    \definecolor{citecolor}{rgb}{.12,.54,.11}

    % ANSI colors
    \definecolor{ansi-black}{HTML}{3E424D}
    \definecolor{ansi-black-intense}{HTML}{282C36}
    \definecolor{ansi-red}{HTML}{E75C58}
    \definecolor{ansi-red-intense}{HTML}{B22B31}
    \definecolor{ansi-green}{HTML}{00A250}
    \definecolor{ansi-green-intense}{HTML}{007427}
    \definecolor{ansi-yellow}{HTML}{DDB62B}
    \definecolor{ansi-yellow-intense}{HTML}{B27D12}
    \definecolor{ansi-blue}{HTML}{208FFB}
    \definecolor{ansi-blue-intense}{HTML}{0065CA}
    \definecolor{ansi-magenta}{HTML}{D160C4}
    \definecolor{ansi-magenta-intense}{HTML}{A03196}
    \definecolor{ansi-cyan}{HTML}{60C6C8}
    \definecolor{ansi-cyan-intense}{HTML}{258F8F}
    \definecolor{ansi-white}{HTML}{C5C1B4}
    \definecolor{ansi-white-intense}{HTML}{A1A6B2}

    % commands and environments needed by pandoc snippets
    % extracted from the output of `pandoc -s`
    \providecommand{\tightlist}{%
      \setlength{\itemsep}{0pt}\setlength{\parskip}{0pt}}
    \DefineVerbatimEnvironment{Highlighting}{Verbatim}{commandchars=\\\{\}}
    % Add ',fontsize=\small' for more characters per line
    \newenvironment{Shaded}{}{}
    \newcommand{\KeywordTok}[1]{\textcolor[rgb]{0.00,0.44,0.13}{\textbf{{#1}}}}
    \newcommand{\DataTypeTok}[1]{\textcolor[rgb]{0.56,0.13,0.00}{{#1}}}
    \newcommand{\DecValTok}[1]{\textcolor[rgb]{0.25,0.63,0.44}{{#1}}}
    \newcommand{\BaseNTok}[1]{\textcolor[rgb]{0.25,0.63,0.44}{{#1}}}
    \newcommand{\FloatTok}[1]{\textcolor[rgb]{0.25,0.63,0.44}{{#1}}}
    \newcommand{\CharTok}[1]{\textcolor[rgb]{0.25,0.44,0.63}{{#1}}}
    \newcommand{\StringTok}[1]{\textcolor[rgb]{0.25,0.44,0.63}{{#1}}}
    \newcommand{\CommentTok}[1]{\textcolor[rgb]{0.38,0.63,0.69}{\textit{{#1}}}}
    \newcommand{\OtherTok}[1]{\textcolor[rgb]{0.00,0.44,0.13}{{#1}}}
    \newcommand{\AlertTok}[1]{\textcolor[rgb]{1.00,0.00,0.00}{\textbf{{#1}}}}
    \newcommand{\FunctionTok}[1]{\textcolor[rgb]{0.02,0.16,0.49}{{#1}}}
    \newcommand{\RegionMarkerTok}[1]{{#1}}
    \newcommand{\ErrorTok}[1]{\textcolor[rgb]{1.00,0.00,0.00}{\textbf{{#1}}}}
    \newcommand{\NormalTok}[1]{{#1}}
    
    % Additional commands for more recent versions of Pandoc
    \newcommand{\ConstantTok}[1]{\textcolor[rgb]{0.53,0.00,0.00}{{#1}}}
    \newcommand{\SpecialCharTok}[1]{\textcolor[rgb]{0.25,0.44,0.63}{{#1}}}
    \newcommand{\VerbatimStringTok}[1]{\textcolor[rgb]{0.25,0.44,0.63}{{#1}}}
    \newcommand{\SpecialStringTok}[1]{\textcolor[rgb]{0.73,0.40,0.53}{{#1}}}
    \newcommand{\ImportTok}[1]{{#1}}
    \newcommand{\DocumentationTok}[1]{\textcolor[rgb]{0.73,0.13,0.13}{\textit{{#1}}}}
    \newcommand{\AnnotationTok}[1]{\textcolor[rgb]{0.38,0.63,0.69}{\textbf{\textit{{#1}}}}}
    \newcommand{\CommentVarTok}[1]{\textcolor[rgb]{0.38,0.63,0.69}{\textbf{\textit{{#1}}}}}
    \newcommand{\VariableTok}[1]{\textcolor[rgb]{0.10,0.09,0.49}{{#1}}}
    \newcommand{\ControlFlowTok}[1]{\textcolor[rgb]{0.00,0.44,0.13}{\textbf{{#1}}}}
    \newcommand{\OperatorTok}[1]{\textcolor[rgb]{0.40,0.40,0.40}{{#1}}}
    \newcommand{\BuiltInTok}[1]{{#1}}
    \newcommand{\ExtensionTok}[1]{{#1}}
    \newcommand{\PreprocessorTok}[1]{\textcolor[rgb]{0.74,0.48,0.00}{{#1}}}
    \newcommand{\AttributeTok}[1]{\textcolor[rgb]{0.49,0.56,0.16}{{#1}}}
    \newcommand{\InformationTok}[1]{\textcolor[rgb]{0.38,0.63,0.69}{\textbf{\textit{{#1}}}}}
    \newcommand{\WarningTok}[1]{\textcolor[rgb]{0.38,0.63,0.69}{\textbf{\textit{{#1}}}}}
    
    
    % Define a nice break command that doesn't care if a line doesn't already
    % exist.
    \def\br{\hspace*{\fill} \\* }
    % Math Jax compatability definitions
    \def\gt{>}
    \def\lt{<}
    % Document parameters
    \title{HW0}
    
    
    

    % Pygments definitions
    
\makeatletter
\def\PY@reset{\let\PY@it=\relax \let\PY@bf=\relax%
    \let\PY@ul=\relax \let\PY@tc=\relax%
    \let\PY@bc=\relax \let\PY@ff=\relax}
\def\PY@tok#1{\csname PY@tok@#1\endcsname}
\def\PY@toks#1+{\ifx\relax#1\empty\else%
    \PY@tok{#1}\expandafter\PY@toks\fi}
\def\PY@do#1{\PY@bc{\PY@tc{\PY@ul{%
    \PY@it{\PY@bf{\PY@ff{#1}}}}}}}
\def\PY#1#2{\PY@reset\PY@toks#1+\relax+\PY@do{#2}}

\expandafter\def\csname PY@tok@w\endcsname{\def\PY@tc##1{\textcolor[rgb]{0.73,0.73,0.73}{##1}}}
\expandafter\def\csname PY@tok@c\endcsname{\let\PY@it=\textit\def\PY@tc##1{\textcolor[rgb]{0.25,0.50,0.50}{##1}}}
\expandafter\def\csname PY@tok@cp\endcsname{\def\PY@tc##1{\textcolor[rgb]{0.74,0.48,0.00}{##1}}}
\expandafter\def\csname PY@tok@k\endcsname{\let\PY@bf=\textbf\def\PY@tc##1{\textcolor[rgb]{0.00,0.50,0.00}{##1}}}
\expandafter\def\csname PY@tok@kp\endcsname{\def\PY@tc##1{\textcolor[rgb]{0.00,0.50,0.00}{##1}}}
\expandafter\def\csname PY@tok@kt\endcsname{\def\PY@tc##1{\textcolor[rgb]{0.69,0.00,0.25}{##1}}}
\expandafter\def\csname PY@tok@o\endcsname{\def\PY@tc##1{\textcolor[rgb]{0.40,0.40,0.40}{##1}}}
\expandafter\def\csname PY@tok@ow\endcsname{\let\PY@bf=\textbf\def\PY@tc##1{\textcolor[rgb]{0.67,0.13,1.00}{##1}}}
\expandafter\def\csname PY@tok@nb\endcsname{\def\PY@tc##1{\textcolor[rgb]{0.00,0.50,0.00}{##1}}}
\expandafter\def\csname PY@tok@nf\endcsname{\def\PY@tc##1{\textcolor[rgb]{0.00,0.00,1.00}{##1}}}
\expandafter\def\csname PY@tok@nc\endcsname{\let\PY@bf=\textbf\def\PY@tc##1{\textcolor[rgb]{0.00,0.00,1.00}{##1}}}
\expandafter\def\csname PY@tok@nn\endcsname{\let\PY@bf=\textbf\def\PY@tc##1{\textcolor[rgb]{0.00,0.00,1.00}{##1}}}
\expandafter\def\csname PY@tok@ne\endcsname{\let\PY@bf=\textbf\def\PY@tc##1{\textcolor[rgb]{0.82,0.25,0.23}{##1}}}
\expandafter\def\csname PY@tok@nv\endcsname{\def\PY@tc##1{\textcolor[rgb]{0.10,0.09,0.49}{##1}}}
\expandafter\def\csname PY@tok@no\endcsname{\def\PY@tc##1{\textcolor[rgb]{0.53,0.00,0.00}{##1}}}
\expandafter\def\csname PY@tok@nl\endcsname{\def\PY@tc##1{\textcolor[rgb]{0.63,0.63,0.00}{##1}}}
\expandafter\def\csname PY@tok@ni\endcsname{\let\PY@bf=\textbf\def\PY@tc##1{\textcolor[rgb]{0.60,0.60,0.60}{##1}}}
\expandafter\def\csname PY@tok@na\endcsname{\def\PY@tc##1{\textcolor[rgb]{0.49,0.56,0.16}{##1}}}
\expandafter\def\csname PY@tok@nt\endcsname{\let\PY@bf=\textbf\def\PY@tc##1{\textcolor[rgb]{0.00,0.50,0.00}{##1}}}
\expandafter\def\csname PY@tok@nd\endcsname{\def\PY@tc##1{\textcolor[rgb]{0.67,0.13,1.00}{##1}}}
\expandafter\def\csname PY@tok@s\endcsname{\def\PY@tc##1{\textcolor[rgb]{0.73,0.13,0.13}{##1}}}
\expandafter\def\csname PY@tok@sd\endcsname{\let\PY@it=\textit\def\PY@tc##1{\textcolor[rgb]{0.73,0.13,0.13}{##1}}}
\expandafter\def\csname PY@tok@si\endcsname{\let\PY@bf=\textbf\def\PY@tc##1{\textcolor[rgb]{0.73,0.40,0.53}{##1}}}
\expandafter\def\csname PY@tok@se\endcsname{\let\PY@bf=\textbf\def\PY@tc##1{\textcolor[rgb]{0.73,0.40,0.13}{##1}}}
\expandafter\def\csname PY@tok@sr\endcsname{\def\PY@tc##1{\textcolor[rgb]{0.73,0.40,0.53}{##1}}}
\expandafter\def\csname PY@tok@ss\endcsname{\def\PY@tc##1{\textcolor[rgb]{0.10,0.09,0.49}{##1}}}
\expandafter\def\csname PY@tok@sx\endcsname{\def\PY@tc##1{\textcolor[rgb]{0.00,0.50,0.00}{##1}}}
\expandafter\def\csname PY@tok@m\endcsname{\def\PY@tc##1{\textcolor[rgb]{0.40,0.40,0.40}{##1}}}
\expandafter\def\csname PY@tok@gh\endcsname{\let\PY@bf=\textbf\def\PY@tc##1{\textcolor[rgb]{0.00,0.00,0.50}{##1}}}
\expandafter\def\csname PY@tok@gu\endcsname{\let\PY@bf=\textbf\def\PY@tc##1{\textcolor[rgb]{0.50,0.00,0.50}{##1}}}
\expandafter\def\csname PY@tok@gd\endcsname{\def\PY@tc##1{\textcolor[rgb]{0.63,0.00,0.00}{##1}}}
\expandafter\def\csname PY@tok@gi\endcsname{\def\PY@tc##1{\textcolor[rgb]{0.00,0.63,0.00}{##1}}}
\expandafter\def\csname PY@tok@gr\endcsname{\def\PY@tc##1{\textcolor[rgb]{1.00,0.00,0.00}{##1}}}
\expandafter\def\csname PY@tok@ge\endcsname{\let\PY@it=\textit}
\expandafter\def\csname PY@tok@gs\endcsname{\let\PY@bf=\textbf}
\expandafter\def\csname PY@tok@gp\endcsname{\let\PY@bf=\textbf\def\PY@tc##1{\textcolor[rgb]{0.00,0.00,0.50}{##1}}}
\expandafter\def\csname PY@tok@go\endcsname{\def\PY@tc##1{\textcolor[rgb]{0.53,0.53,0.53}{##1}}}
\expandafter\def\csname PY@tok@gt\endcsname{\def\PY@tc##1{\textcolor[rgb]{0.00,0.27,0.87}{##1}}}
\expandafter\def\csname PY@tok@err\endcsname{\def\PY@bc##1{\setlength{\fboxsep}{0pt}\fcolorbox[rgb]{1.00,0.00,0.00}{1,1,1}{\strut ##1}}}
\expandafter\def\csname PY@tok@kc\endcsname{\let\PY@bf=\textbf\def\PY@tc##1{\textcolor[rgb]{0.00,0.50,0.00}{##1}}}
\expandafter\def\csname PY@tok@kd\endcsname{\let\PY@bf=\textbf\def\PY@tc##1{\textcolor[rgb]{0.00,0.50,0.00}{##1}}}
\expandafter\def\csname PY@tok@kn\endcsname{\let\PY@bf=\textbf\def\PY@tc##1{\textcolor[rgb]{0.00,0.50,0.00}{##1}}}
\expandafter\def\csname PY@tok@kr\endcsname{\let\PY@bf=\textbf\def\PY@tc##1{\textcolor[rgb]{0.00,0.50,0.00}{##1}}}
\expandafter\def\csname PY@tok@bp\endcsname{\def\PY@tc##1{\textcolor[rgb]{0.00,0.50,0.00}{##1}}}
\expandafter\def\csname PY@tok@fm\endcsname{\def\PY@tc##1{\textcolor[rgb]{0.00,0.00,1.00}{##1}}}
\expandafter\def\csname PY@tok@vc\endcsname{\def\PY@tc##1{\textcolor[rgb]{0.10,0.09,0.49}{##1}}}
\expandafter\def\csname PY@tok@vg\endcsname{\def\PY@tc##1{\textcolor[rgb]{0.10,0.09,0.49}{##1}}}
\expandafter\def\csname PY@tok@vi\endcsname{\def\PY@tc##1{\textcolor[rgb]{0.10,0.09,0.49}{##1}}}
\expandafter\def\csname PY@tok@vm\endcsname{\def\PY@tc##1{\textcolor[rgb]{0.10,0.09,0.49}{##1}}}
\expandafter\def\csname PY@tok@sa\endcsname{\def\PY@tc##1{\textcolor[rgb]{0.73,0.13,0.13}{##1}}}
\expandafter\def\csname PY@tok@sb\endcsname{\def\PY@tc##1{\textcolor[rgb]{0.73,0.13,0.13}{##1}}}
\expandafter\def\csname PY@tok@sc\endcsname{\def\PY@tc##1{\textcolor[rgb]{0.73,0.13,0.13}{##1}}}
\expandafter\def\csname PY@tok@dl\endcsname{\def\PY@tc##1{\textcolor[rgb]{0.73,0.13,0.13}{##1}}}
\expandafter\def\csname PY@tok@s2\endcsname{\def\PY@tc##1{\textcolor[rgb]{0.73,0.13,0.13}{##1}}}
\expandafter\def\csname PY@tok@sh\endcsname{\def\PY@tc##1{\textcolor[rgb]{0.73,0.13,0.13}{##1}}}
\expandafter\def\csname PY@tok@s1\endcsname{\def\PY@tc##1{\textcolor[rgb]{0.73,0.13,0.13}{##1}}}
\expandafter\def\csname PY@tok@mb\endcsname{\def\PY@tc##1{\textcolor[rgb]{0.40,0.40,0.40}{##1}}}
\expandafter\def\csname PY@tok@mf\endcsname{\def\PY@tc##1{\textcolor[rgb]{0.40,0.40,0.40}{##1}}}
\expandafter\def\csname PY@tok@mh\endcsname{\def\PY@tc##1{\textcolor[rgb]{0.40,0.40,0.40}{##1}}}
\expandafter\def\csname PY@tok@mi\endcsname{\def\PY@tc##1{\textcolor[rgb]{0.40,0.40,0.40}{##1}}}
\expandafter\def\csname PY@tok@il\endcsname{\def\PY@tc##1{\textcolor[rgb]{0.40,0.40,0.40}{##1}}}
\expandafter\def\csname PY@tok@mo\endcsname{\def\PY@tc##1{\textcolor[rgb]{0.40,0.40,0.40}{##1}}}
\expandafter\def\csname PY@tok@ch\endcsname{\let\PY@it=\textit\def\PY@tc##1{\textcolor[rgb]{0.25,0.50,0.50}{##1}}}
\expandafter\def\csname PY@tok@cm\endcsname{\let\PY@it=\textit\def\PY@tc##1{\textcolor[rgb]{0.25,0.50,0.50}{##1}}}
\expandafter\def\csname PY@tok@cpf\endcsname{\let\PY@it=\textit\def\PY@tc##1{\textcolor[rgb]{0.25,0.50,0.50}{##1}}}
\expandafter\def\csname PY@tok@c1\endcsname{\let\PY@it=\textit\def\PY@tc##1{\textcolor[rgb]{0.25,0.50,0.50}{##1}}}
\expandafter\def\csname PY@tok@cs\endcsname{\let\PY@it=\textit\def\PY@tc##1{\textcolor[rgb]{0.25,0.50,0.50}{##1}}}

\def\PYZbs{\char`\\}
\def\PYZus{\char`\_}
\def\PYZob{\char`\{}
\def\PYZcb{\char`\}}
\def\PYZca{\char`\^}
\def\PYZam{\char`\&}
\def\PYZlt{\char`\<}
\def\PYZgt{\char`\>}
\def\PYZsh{\char`\#}
\def\PYZpc{\char`\%}
\def\PYZdl{\char`\$}
\def\PYZhy{\char`\-}
\def\PYZsq{\char`\'}
\def\PYZdq{\char`\"}
\def\PYZti{\char`\~}
% for compatibility with earlier versions
\def\PYZat{@}
\def\PYZlb{[}
\def\PYZrb{]}
\makeatother


    % Exact colors from NB
    \definecolor{incolor}{rgb}{0.0, 0.0, 0.5}
    \definecolor{outcolor}{rgb}{0.545, 0.0, 0.0}



    
    % Prevent overflowing lines due to hard-to-break entities
    \sloppy 
    % Setup hyperref package
    \hypersetup{
      breaklinks=true,  % so long urls are correctly broken across lines
      colorlinks=true,
      urlcolor=urlcolor,
      linkcolor=linkcolor,
      citecolor=citecolor,
      }
    % Slightly bigger margins than the latex defaults
    
    \geometry{verbose,tmargin=1in,bmargin=1in,lmargin=1in,rmargin=1in}
    
    

    \begin{document}
    
    
    \maketitle
    
    

    
    \section{CSE 252A Computer Vision I Fall 2018 - Assignment
0}\label{cse-252a-computer-vision-i-fall-2018---assignment-0}

\subsubsection{Instructor: David
Kriegman}\label{instructor-david-kriegman}

\subsubsection{Assignment Published On: Tuesday, October 2,
2018}\label{assignment-published-on-tuesday-october-2-2018}

\subsubsection{Due On: Tuesday, October 9, 2018 11:59
pm}\label{due-on-tuesday-october-9-2018-1159-pm}

\subsection{Instructions}\label{instructions}

\begin{itemize}
\item
  Review the academic integrity and collaboration policies on the course
  website.
\item
  This assignment must be completed individually.
\item
  All solutions must be written in this notebook
\item
  Programming aspects of this assignment must be completed using Python
  in this notebook.
\item
  If you want to modify the skeleton code, you can do so. This has been
  provided just to provide you with a framework for the solution.
\item
  You may use python packages for basic linear algebra (you can use
  numpy or scipy for basic operations), but you may not use packages
  that directly solve the problem.
\item
  If you are unsure about using a specific package or function, then ask
  the instructor and teaching assistants for clarification.
\item
  You must submit this notebook exported as a pdf. You must also submit
  this notebook as .ipynb file.
\item
  You must submit both files (.pdf and .ipynb) on Gradescope. You must
  mark each problem on Gradescope in the pdf.
\item
  It is highly recommended that you begin working on this assignment
  early.
\item ~
  \subsection{\texorpdfstring{\textbf{Late policy} - 10\% per day late
  penalty after due
  date.}{Late policy - 10\% per day late penalty after due date.}}\label{late-policy---10-per-day-late-penalty-after-due-date.}
\end{itemize}

Welcome to CSE252A Computer Vision I! This course gives you a
comprehensive introduction to computer vison providing broad coverage
including low level vision, inferring 3D properties from images, and
object recognition. We will be using a variety of tools in this class
that will require some initial configuration. To ensure smooth progress,
we will setup the majority of the tools to be used in this course in
this assignment. You will also practice some basic image manipulation
techniques. Finally, you will need to export this Ipython notebook as
pdf and submit it to Gradescope along with .ipynb file before the due
date.

\subsubsection{Piazza, Gradescope and
Python}\label{piazza-gradescope-and-python}

\textbf{Piazza}

Go to \href{https://piazza.com/ucsd/fall2018/cse252a}{Piazza} and sign
up for the class using your ucsd.edu email account. You'll be able to
ask the professor, the TAs and your classmates questions on Piazza.
Class announcements will be made using Piazza, so make sure you check
your email or Piazza frequently.

\textbf{Gradescope}

Every student will get an email regarding gradescope signup once
enrolled in this class. All the assignments are required to be submitted
to gradescope for grading. Make sure that you mark each page for
different problems.

\textbf{Python}

We will use the Python programming language for all assignments in this
course, with a few popular libraries (numpy, matplotlib). Assignments
will be given in the format of browser-based Jupyter/Ipython notebook
that you are currently viewing. We expect that many of you have some
experience with Python and Numpy. And if you have previous knowledge in
Matlab, check out the
\href{https://docs.scipy.org/doc/numpy-dev/user/numpy-for-matlab-users.html}{numpy
for Matlab users} page. The section below will serve as a quick
introduction to Numpy and some other libraries.

    \subsection{Getting started with
Numpy}\label{getting-started-with-numpy}

Numpy is the fundamental package for scientific computing with Python.
It provides a powerful N-dimensional array object and functions for
working with these arrays.

    \subsubsection{Arrays}\label{arrays}

    \begin{Verbatim}[commandchars=\\\{\}]
{\color{incolor}In [{\color{incolor}1}]:} \PY{k+kn}{import} \PY{n+nn}{numpy} \PY{k}{as} \PY{n+nn}{np}
        
        \PY{n}{v} \PY{o}{=} \PY{n}{np}\PY{o}{.}\PY{n}{array}\PY{p}{(}\PY{p}{[}\PY{l+m+mi}{1}\PY{p}{,} \PY{l+m+mi}{0}\PY{p}{,} \PY{l+m+mi}{0}\PY{p}{]}\PY{p}{)}        \PY{c+c1}{\PYZsh{} a 1d array}
        \PY{n+nb}{print}\PY{p}{(}\PY{l+s+s2}{\PYZdq{}}\PY{l+s+s2}{1d array}\PY{l+s+s2}{\PYZdq{}}\PY{p}{)}
        \PY{n+nb}{print}\PY{p}{(}\PY{n}{v}\PY{p}{)}
        \PY{n+nb}{print}\PY{p}{(}\PY{n}{v}\PY{o}{.}\PY{n}{shape}\PY{p}{)}                 \PY{c+c1}{\PYZsh{} print the size of v}
        \PY{n}{v} \PY{o}{=} \PY{n}{np}\PY{o}{.}\PY{n}{array}\PY{p}{(}\PY{p}{[}\PY{p}{[}\PY{l+m+mi}{1}\PY{p}{]}\PY{p}{,} \PY{p}{[}\PY{l+m+mi}{2}\PY{p}{]}\PY{p}{,} \PY{p}{[}\PY{l+m+mi}{3}\PY{p}{]}\PY{p}{]}\PY{p}{)}  \PY{c+c1}{\PYZsh{} a 2d array}
        \PY{n+nb}{print}\PY{p}{(}\PY{l+s+s2}{\PYZdq{}}\PY{l+s+se}{\PYZbs{}n}\PY{l+s+s2}{2d array}\PY{l+s+s2}{\PYZdq{}}\PY{p}{)}
        \PY{n+nb}{print}\PY{p}{(}\PY{n}{v}\PY{p}{)}
        \PY{n+nb}{print}\PY{p}{(}\PY{n}{v}\PY{o}{.}\PY{n}{shape}\PY{p}{)}                 \PY{c+c1}{\PYZsh{} print the size of v, notice the difference}
        \PY{n}{v} \PY{o}{=} \PY{n}{v}\PY{o}{.}\PY{n}{T}                        \PY{c+c1}{\PYZsh{} transpose of a 2d array}
        
        \PY{n}{m} \PY{o}{=} \PY{n}{np}\PY{o}{.}\PY{n}{zeros}\PY{p}{(}\PY{p}{[}\PY{l+m+mi}{2}\PY{p}{,} \PY{l+m+mi}{3}\PY{p}{]}\PY{p}{)}           \PY{c+c1}{\PYZsh{} a 2x3 array of zeros}
        \PY{n}{v} \PY{o}{=} \PY{n}{np}\PY{o}{.}\PY{n}{ones}\PY{p}{(}\PY{p}{[}\PY{l+m+mi}{1}\PY{p}{,} \PY{l+m+mi}{3}\PY{p}{]}\PY{p}{)}            \PY{c+c1}{\PYZsh{} a 1x3 array of ones}
        \PY{n}{m} \PY{o}{=} \PY{n}{np}\PY{o}{.}\PY{n}{eye}\PY{p}{(}\PY{l+m+mi}{3}\PY{p}{)}                  \PY{c+c1}{\PYZsh{} identity matrix}
        \PY{n}{v} \PY{o}{=} \PY{n}{np}\PY{o}{.}\PY{n}{random}\PY{o}{.}\PY{n}{rand}\PY{p}{(}\PY{l+m+mi}{3}\PY{p}{,} \PY{l+m+mi}{1}\PY{p}{)}       \PY{c+c1}{\PYZsh{} random matrix with values in [0, 1]}
        \PY{n}{m} \PY{o}{=} \PY{n}{np}\PY{o}{.}\PY{n}{ones}\PY{p}{(}\PY{n}{v}\PY{o}{.}\PY{n}{shape}\PY{p}{)} \PY{o}{*} \PY{l+m+mi}{3}       \PY{c+c1}{\PYZsh{} create a matrix from shape}
\end{Verbatim}


    \begin{Verbatim}[commandchars=\\\{\}]
1d array
[1 0 0]
(3,)

2d array
[[1]
 [2]
 [3]]
(3, 1)

    \end{Verbatim}

    \subsubsection{Array indexing}\label{array-indexing}

    \begin{Verbatim}[commandchars=\\\{\}]
{\color{incolor}In [{\color{incolor}2}]:} \PY{k+kn}{import} \PY{n+nn}{numpy} \PY{k}{as} \PY{n+nn}{np}
        
        \PY{n}{m} \PY{o}{=} \PY{n}{np}\PY{o}{.}\PY{n}{array}\PY{p}{(}\PY{p}{[}\PY{p}{[}\PY{l+m+mi}{1}\PY{p}{,} \PY{l+m+mi}{2}\PY{p}{,} \PY{l+m+mi}{3}\PY{p}{]}\PY{p}{,} \PY{p}{[}\PY{l+m+mi}{4}\PY{p}{,} \PY{l+m+mi}{5}\PY{p}{,} \PY{l+m+mi}{6}\PY{p}{]}\PY{p}{]}\PY{p}{)}\PY{c+c1}{\PYZsh{} create a 2d array with shape (2, 3)}
        \PY{n+nb}{print}\PY{p}{(}\PY{l+s+s2}{\PYZdq{}}\PY{l+s+s2}{Access a single element}\PY{l+s+s2}{\PYZdq{}}\PY{p}{)}
        \PY{n+nb}{print}\PY{p}{(}\PY{n}{m}\PY{p}{[}\PY{l+m+mi}{0}\PY{p}{,} \PY{l+m+mi}{2}\PY{p}{]}\PY{p}{)}                  \PY{c+c1}{\PYZsh{} access an element}
        \PY{n}{m}\PY{p}{[}\PY{l+m+mi}{0}\PY{p}{,} \PY{l+m+mi}{2}\PY{p}{]} \PY{o}{=} \PY{l+m+mi}{252}                   \PY{c+c1}{\PYZsh{} a slice of an array is a view into the same data;}
        \PY{n+nb}{print}\PY{p}{(}\PY{l+s+s2}{\PYZdq{}}\PY{l+s+se}{\PYZbs{}n}\PY{l+s+s2}{Modified a single element}\PY{l+s+s2}{\PYZdq{}}\PY{p}{)}
        \PY{n+nb}{print}\PY{p}{(}\PY{n}{m}\PY{p}{)}                        \PY{c+c1}{\PYZsh{} this will modify the original array}
        
        \PY{n+nb}{print}\PY{p}{(}\PY{l+s+s2}{\PYZdq{}}\PY{l+s+se}{\PYZbs{}n}\PY{l+s+s2}{Access a subarray}\PY{l+s+s2}{\PYZdq{}}\PY{p}{)}
        \PY{n+nb}{print}\PY{p}{(}\PY{n}{m}\PY{p}{[}\PY{l+m+mi}{1}\PY{p}{,} \PY{p}{:}\PY{p}{]}\PY{p}{)}                  \PY{c+c1}{\PYZsh{} access a row (to 1d array)}
        \PY{n+nb}{print}\PY{p}{(}\PY{n}{m}\PY{p}{[}\PY{l+m+mi}{1}\PY{p}{:}\PY{p}{,} \PY{p}{:}\PY{p}{]}\PY{p}{)}                 \PY{c+c1}{\PYZsh{} access a row (to 2d array)}
        \PY{n+nb}{print}\PY{p}{(}\PY{l+s+s2}{\PYZdq{}}\PY{l+s+se}{\PYZbs{}n}\PY{l+s+s2}{Transpose a subarray}\PY{l+s+s2}{\PYZdq{}}\PY{p}{)}
        \PY{n+nb}{print}\PY{p}{(}\PY{n}{m}\PY{p}{[}\PY{l+m+mi}{1}\PY{p}{,} \PY{p}{:}\PY{p}{]}\PY{o}{.}\PY{n}{T}\PY{p}{)}        \PY{c+c1}{\PYZsh{} notice the difference of the dimension of resulting array}
        \PY{n+nb}{print}\PY{p}{(}\PY{n}{m}\PY{p}{[}\PY{l+m+mi}{1}\PY{p}{:}\PY{p}{,} \PY{p}{:}\PY{p}{]}\PY{o}{.}\PY{n}{T}\PY{p}{)}       \PY{c+c1}{\PYZsh{} this will be helpful if you want to transpose it later}
        
        \PY{c+c1}{\PYZsh{} Boolean array indexing}
        \PY{c+c1}{\PYZsh{} Given a array m, create a new array with values equal to m }
        \PY{c+c1}{\PYZsh{} if they are greater than 0, and equal to 0 if they less than or equal 0}
        
        \PY{n}{m} \PY{o}{=} \PY{n}{np}\PY{o}{.}\PY{n}{array}\PY{p}{(}\PY{p}{[}\PY{p}{[}\PY{l+m+mi}{3}\PY{p}{,} \PY{l+m+mi}{5}\PY{p}{,} \PY{o}{\PYZhy{}}\PY{l+m+mi}{2}\PY{p}{]}\PY{p}{,} \PY{p}{[}\PY{l+m+mi}{5}\PY{p}{,} \PY{o}{\PYZhy{}}\PY{l+m+mi}{1}\PY{p}{,} \PY{l+m+mi}{0}\PY{p}{]}\PY{p}{]}\PY{p}{)}
        \PY{n}{n} \PY{o}{=} \PY{n}{np}\PY{o}{.}\PY{n}{zeros}\PY{p}{(}\PY{n}{m}\PY{o}{.}\PY{n}{shape}\PY{p}{)}
        \PY{n}{n}\PY{p}{[}\PY{n}{m} \PY{o}{\PYZgt{}} \PY{l+m+mi}{0}\PY{p}{]} \PY{o}{=} \PY{n}{m}\PY{p}{[}\PY{n}{m} \PY{o}{\PYZgt{}} \PY{l+m+mi}{0}\PY{p}{]}
        \PY{n+nb}{print}\PY{p}{(}\PY{l+s+s2}{\PYZdq{}}\PY{l+s+se}{\PYZbs{}n}\PY{l+s+s2}{Boolean array indexing}\PY{l+s+s2}{\PYZdq{}}\PY{p}{)}
        \PY{n+nb}{print}\PY{p}{(}\PY{n}{n}\PY{p}{)}
\end{Verbatim}


    \begin{Verbatim}[commandchars=\\\{\}]
Access a single element
3

Modified a single element
[[  1   2 252]
 [  4   5   6]]

Access a subarray
[4 5 6]
[[4 5 6]]

Transpose a subarray
[4 5 6]
[[4]
 [5]
 [6]]

Boolean array indexing
[[3. 5. 0.]
 [5. 0. 0.]]

    \end{Verbatim}

    \subsubsection{Operations on array}\label{operations-on-array}

\textbf{Elementwise Operations}

    \begin{Verbatim}[commandchars=\\\{\}]
{\color{incolor}In [{\color{incolor}3}]:} \PY{k+kn}{import} \PY{n+nn}{numpy} \PY{k}{as} \PY{n+nn}{np}
        
        \PY{n}{a} \PY{o}{=} \PY{n}{np}\PY{o}{.}\PY{n}{array}\PY{p}{(}\PY{p}{[}\PY{p}{[}\PY{l+m+mi}{1}\PY{p}{,} \PY{l+m+mi}{2}\PY{p}{,} \PY{l+m+mi}{3}\PY{p}{]}\PY{p}{,} \PY{p}{[}\PY{l+m+mi}{2}\PY{p}{,} \PY{l+m+mi}{3}\PY{p}{,} \PY{l+m+mi}{4}\PY{p}{]}\PY{p}{]}\PY{p}{,} \PY{n}{dtype}\PY{o}{=}\PY{n}{np}\PY{o}{.}\PY{n}{float64}\PY{p}{)}
        \PY{n+nb}{print}\PY{p}{(}\PY{n}{a} \PY{o}{*} \PY{l+m+mi}{2}\PY{p}{)}                                            \PY{c+c1}{\PYZsh{} scalar multiplication}
        \PY{n+nb}{print}\PY{p}{(}\PY{n}{a} \PY{o}{/} \PY{l+m+mi}{4}\PY{p}{)}                                            \PY{c+c1}{\PYZsh{} scalar division}
        \PY{n+nb}{print}\PY{p}{(}\PY{n}{np}\PY{o}{.}\PY{n}{round}\PY{p}{(}\PY{n}{a} \PY{o}{/} \PY{l+m+mi}{4}\PY{p}{)}\PY{p}{)}
        \PY{n+nb}{print}\PY{p}{(}\PY{n}{np}\PY{o}{.}\PY{n}{power}\PY{p}{(}\PY{n}{a}\PY{p}{,} \PY{l+m+mi}{2}\PY{p}{)}\PY{p}{)}
        \PY{n+nb}{print}\PY{p}{(}\PY{n}{np}\PY{o}{.}\PY{n}{log}\PY{p}{(}\PY{n}{a}\PY{p}{)}\PY{p}{)}
        
        
        \PY{n}{b} \PY{o}{=} \PY{n}{np}\PY{o}{.}\PY{n}{array}\PY{p}{(}\PY{p}{[}\PY{p}{[}\PY{l+m+mi}{5}\PY{p}{,} \PY{l+m+mi}{6}\PY{p}{,} \PY{l+m+mi}{7}\PY{p}{]}\PY{p}{,} \PY{p}{[}\PY{l+m+mi}{5}\PY{p}{,} \PY{l+m+mi}{7}\PY{p}{,} \PY{l+m+mi}{8}\PY{p}{]}\PY{p}{]}\PY{p}{,} \PY{n}{dtype}\PY{o}{=}\PY{n}{np}\PY{o}{.}\PY{n}{float64}\PY{p}{)}
        \PY{n+nb}{print}\PY{p}{(}\PY{n}{a} \PY{o}{+} \PY{n}{b}\PY{p}{)}                                            \PY{c+c1}{\PYZsh{} elementwise sum}
        \PY{n+nb}{print}\PY{p}{(}\PY{n}{a} \PY{o}{\PYZhy{}} \PY{n}{b}\PY{p}{)}                                            \PY{c+c1}{\PYZsh{} elementwise difference}
        \PY{n+nb}{print}\PY{p}{(}\PY{n}{a} \PY{o}{*} \PY{n}{b}\PY{p}{)}                                            \PY{c+c1}{\PYZsh{} elementwise product}
        \PY{n+nb}{print}\PY{p}{(}\PY{n}{a} \PY{o}{/} \PY{n}{b}\PY{p}{)}                                            \PY{c+c1}{\PYZsh{} elementwise division}
\end{Verbatim}


    \begin{Verbatim}[commandchars=\\\{\}]
[[2. 4. 6.]
 [4. 6. 8.]]
[[0.25 0.5  0.75]
 [0.5  0.75 1.  ]]
[[0. 0. 1.]
 [0. 1. 1.]]
[[ 1.  4.  9.]
 [ 4.  9. 16.]]
[[0.         0.69314718 1.09861229]
 [0.69314718 1.09861229 1.38629436]]
[[ 6.  8. 10.]
 [ 7. 10. 12.]]
[[-4. -4. -4.]
 [-3. -4. -4.]]
[[ 5. 12. 21.]
 [10. 21. 32.]]
[[0.2        0.33333333 0.42857143]
 [0.4        0.42857143 0.5       ]]

    \end{Verbatim}

    \textbf{Vector Operations}

    \begin{Verbatim}[commandchars=\\\{\}]
{\color{incolor}In [{\color{incolor}4}]:} \PY{k+kn}{import} \PY{n+nn}{numpy} \PY{k}{as} \PY{n+nn}{np}
        
        \PY{n}{a} \PY{o}{=} \PY{n}{np}\PY{o}{.}\PY{n}{array}\PY{p}{(}\PY{p}{[}\PY{p}{[}\PY{l+m+mi}{1}\PY{p}{,} \PY{l+m+mi}{2}\PY{p}{]}\PY{p}{,} \PY{p}{[}\PY{l+m+mi}{3}\PY{p}{,} \PY{l+m+mi}{4}\PY{p}{]}\PY{p}{]}\PY{p}{)}
        \PY{n+nb}{print}\PY{p}{(}\PY{l+s+s2}{\PYZdq{}}\PY{l+s+s2}{sum of array}\PY{l+s+s2}{\PYZdq{}}\PY{p}{)}
        \PY{n+nb}{print}\PY{p}{(}\PY{n}{np}\PY{o}{.}\PY{n}{sum}\PY{p}{(}\PY{n}{a}\PY{p}{)}\PY{p}{)}                \PY{c+c1}{\PYZsh{} sum of all array elements}
        \PY{n+nb}{print}\PY{p}{(}\PY{n}{np}\PY{o}{.}\PY{n}{sum}\PY{p}{(}\PY{n}{a}\PY{p}{,} \PY{n}{axis}\PY{o}{=}\PY{l+m+mi}{0}\PY{p}{)}\PY{p}{)}        \PY{c+c1}{\PYZsh{} sum of each column}
        \PY{n+nb}{print}\PY{p}{(}\PY{n}{np}\PY{o}{.}\PY{n}{sum}\PY{p}{(}\PY{n}{a}\PY{p}{,} \PY{n}{axis}\PY{o}{=}\PY{l+m+mi}{1}\PY{p}{)}\PY{p}{)}        \PY{c+c1}{\PYZsh{} sum of each row}
        \PY{n+nb}{print}\PY{p}{(}\PY{l+s+s2}{\PYZdq{}}\PY{l+s+se}{\PYZbs{}n}\PY{l+s+s2}{mean of array}\PY{l+s+s2}{\PYZdq{}}\PY{p}{)}
        \PY{n+nb}{print}\PY{p}{(}\PY{n}{np}\PY{o}{.}\PY{n}{mean}\PY{p}{(}\PY{n}{a}\PY{p}{)}\PY{p}{)}               \PY{c+c1}{\PYZsh{} mean of all array elements}
        \PY{n+nb}{print}\PY{p}{(}\PY{n}{np}\PY{o}{.}\PY{n}{mean}\PY{p}{(}\PY{n}{a}\PY{p}{,} \PY{n}{axis}\PY{o}{=}\PY{l+m+mi}{0}\PY{p}{)}\PY{p}{)}       \PY{c+c1}{\PYZsh{} mean of each column}
        \PY{n+nb}{print}\PY{p}{(}\PY{n}{np}\PY{o}{.}\PY{n}{mean}\PY{p}{(}\PY{n}{a}\PY{p}{,} \PY{n}{axis}\PY{o}{=}\PY{l+m+mi}{1}\PY{p}{)}\PY{p}{)}       \PY{c+c1}{\PYZsh{} mean of each row}
\end{Verbatim}


    \begin{Verbatim}[commandchars=\\\{\}]
sum of array
10
[4 6]
[3 7]

mean of array
2.5
[2. 3.]
[1.5 3.5]

    \end{Verbatim}

    \textbf{Matrix Operations}

    \begin{Verbatim}[commandchars=\\\{\}]
{\color{incolor}In [{\color{incolor}5}]:} \PY{k+kn}{import} \PY{n+nn}{numpy} \PY{k}{as} \PY{n+nn}{np}
        
        \PY{n}{a} \PY{o}{=} \PY{n}{np}\PY{o}{.}\PY{n}{array}\PY{p}{(}\PY{p}{[}\PY{p}{[}\PY{l+m+mi}{1}\PY{p}{,} \PY{l+m+mi}{2}\PY{p}{]}\PY{p}{,} \PY{p}{[}\PY{l+m+mi}{3}\PY{p}{,} \PY{l+m+mi}{4}\PY{p}{]}\PY{p}{]}\PY{p}{)}
        \PY{n}{b} \PY{o}{=} \PY{n}{np}\PY{o}{.}\PY{n}{array}\PY{p}{(}\PY{p}{[}\PY{p}{[}\PY{l+m+mi}{5}\PY{p}{,} \PY{l+m+mi}{6}\PY{p}{]}\PY{p}{,} \PY{p}{[}\PY{l+m+mi}{7}\PY{p}{,} \PY{l+m+mi}{8}\PY{p}{]}\PY{p}{]}\PY{p}{)}
        \PY{n+nb}{print}\PY{p}{(}\PY{l+s+s2}{\PYZdq{}}\PY{l+s+s2}{matrix\PYZhy{}matrix product}\PY{l+s+s2}{\PYZdq{}}\PY{p}{)}
        \PY{n+nb}{print}\PY{p}{(}\PY{n}{a}\PY{o}{.}\PY{n}{dot}\PY{p}{(}\PY{n}{b}\PY{p}{)}\PY{p}{)}                 \PY{c+c1}{\PYZsh{} matrix product}
        \PY{n+nb}{print}\PY{p}{(}\PY{n}{a}\PY{o}{.}\PY{n}{T}\PY{o}{.}\PY{n}{dot}\PY{p}{(}\PY{n}{b}\PY{o}{.}\PY{n}{T}\PY{p}{)}\PY{p}{)}
        
        \PY{n}{x} \PY{o}{=} \PY{n}{np}\PY{o}{.}\PY{n}{array}\PY{p}{(}\PY{p}{[}\PY{l+m+mi}{1}\PY{p}{,} \PY{l+m+mi}{2}\PY{p}{]}\PY{p}{)}  
        \PY{n+nb}{print}\PY{p}{(}\PY{l+s+s2}{\PYZdq{}}\PY{l+s+se}{\PYZbs{}n}\PY{l+s+s2}{matrix\PYZhy{}vector product}\PY{l+s+s2}{\PYZdq{}}\PY{p}{)}
        \PY{n+nb}{print}\PY{p}{(}\PY{n}{a}\PY{o}{.}\PY{n}{dot}\PY{p}{(}\PY{n}{x}\PY{p}{)}\PY{p}{)}                 \PY{c+c1}{\PYZsh{} matrix / vector product}
\end{Verbatim}


    \begin{Verbatim}[commandchars=\\\{\}]
matrix-matrix product
[[19 22]
 [43 50]]
[[23 31]
 [34 46]]

matrix-vector product
[ 5 11]

    \end{Verbatim}

    \subsubsection{Matplotlib}\label{matplotlib}

Matplotlib is a plotting library. We will use it to show the result in
this assignment.

    \begin{Verbatim}[commandchars=\\\{\}]
{\color{incolor}In [{\color{incolor}6}]:} \PY{c+c1}{\PYZsh{} this line prepares IPython for working with matplotlib}
        \PY{o}{\PYZpc{}}\PY{k}{matplotlib} inline  
        
        \PY{k+kn}{import} \PY{n+nn}{numpy} \PY{k}{as} \PY{n+nn}{np}
        \PY{k+kn}{import} \PY{n+nn}{matplotlib}\PY{n+nn}{.}\PY{n+nn}{pyplot} \PY{k}{as} \PY{n+nn}{plt}
        \PY{k+kn}{import} \PY{n+nn}{math}
        
        \PY{n}{x} \PY{o}{=} \PY{n}{np}\PY{o}{.}\PY{n}{arange}\PY{p}{(}\PY{o}{\PYZhy{}}\PY{l+m+mi}{24}\PY{p}{,} \PY{l+m+mi}{24}\PY{p}{)} \PY{o}{/} \PY{l+m+mf}{24.} \PY{o}{*} \PY{n}{math}\PY{o}{.}\PY{n}{pi}
        \PY{n}{plt}\PY{o}{.}\PY{n}{plot}\PY{p}{(}\PY{n}{x}\PY{p}{,} \PY{n}{np}\PY{o}{.}\PY{n}{sin}\PY{p}{(}\PY{n}{x}\PY{p}{)}\PY{p}{)}
        \PY{n}{plt}\PY{o}{.}\PY{n}{xlabel}\PY{p}{(}\PY{l+s+s1}{\PYZsq{}}\PY{l+s+s1}{radians}\PY{l+s+s1}{\PYZsq{}}\PY{p}{)}
        \PY{n}{plt}\PY{o}{.}\PY{n}{ylabel}\PY{p}{(}\PY{l+s+s1}{\PYZsq{}}\PY{l+s+s1}{sin value}\PY{l+s+s1}{\PYZsq{}}\PY{p}{)}
        \PY{n}{plt}\PY{o}{.}\PY{n}{title}\PY{p}{(}\PY{l+s+s1}{\PYZsq{}}\PY{l+s+s1}{dummy}\PY{l+s+s1}{\PYZsq{}}\PY{p}{)}
        
        \PY{n}{plt}\PY{o}{.}\PY{n}{show}\PY{p}{(}\PY{p}{)}
\end{Verbatim}


    \begin{center}
    \adjustimage{max size={0.9\linewidth}{0.9\paperheight}}{output_13_0.png}
    \end{center}
    { \hspace*{\fill} \\}
    
    This breif overview introduces many basic functions from a few popular
libraries, but is far from complete. Check out the documentations for
\href{https://docs.scipy.org/doc/numpy/reference/}{Numpy} and
\href{https://matplotlib.org/}{Matplotlib} to find out more.

\begin{center}\rule{0.5\linewidth}{\linethickness}\end{center}

    \subsection{Problem 1 Image operations and vectorization
(1pt)}\label{problem-1-image-operations-and-vectorization-1pt}

Vector operations using numpy can offer a significant speedup over doing
an operation iteratively on an image. The problem below will demonstrate
the time it takes for both approaches to change the color of quadrants
of an image.

The problem reads an image "Lenna.png" that you will find in the
assignment folder. Two functions are then provided as different
approaches for doing an operation on the image.

Your task is to follow through the code and fill in the "piazza"
function using instructions on Piazza.

    \begin{Verbatim}[commandchars=\\\{\}]
{\color{incolor}In [{\color{incolor}7}]:} \PY{k+kn}{import} \PY{n+nn}{numpy} \PY{k}{as} \PY{n+nn}{np}
        \PY{k+kn}{import} \PY{n+nn}{matplotlib}\PY{n+nn}{.}\PY{n+nn}{pyplot} \PY{k}{as} \PY{n+nn}{plt}
        \PY{k+kn}{import} \PY{n+nn}{copy}
        \PY{k+kn}{import} \PY{n+nn}{time}
        
        \PY{n}{img} \PY{o}{=} \PY{n}{plt}\PY{o}{.}\PY{n}{imread}\PY{p}{(}\PY{l+s+s1}{\PYZsq{}}\PY{l+s+s1}{Lenna.png}\PY{l+s+s1}{\PYZsq{}}\PY{p}{)}             \PY{c+c1}{\PYZsh{} read a JPEG image }
        \PY{n+nb}{print}\PY{p}{(}\PY{l+s+s2}{\PYZdq{}}\PY{l+s+s2}{Image shape}\PY{l+s+s2}{\PYZdq{}}\PY{p}{,} \PY{n}{img}\PY{o}{.}\PY{n}{shape}\PY{p}{)}           \PY{c+c1}{\PYZsh{} print image size and color depth}
        
        \PY{n}{plt}\PY{o}{.}\PY{n}{imshow}\PY{p}{(}\PY{n}{img}\PY{p}{)}                           \PY{c+c1}{\PYZsh{} displaying the original image}
        \PY{n}{plt}\PY{o}{.}\PY{n}{show}\PY{p}{(}\PY{p}{)}
\end{Verbatim}


    \begin{Verbatim}[commandchars=\\\{\}]
Image shape (512, 512, 3)

    \end{Verbatim}

    \begin{center}
    \adjustimage{max size={0.9\linewidth}{0.9\paperheight}}{output_16_1.png}
    \end{center}
    { \hspace*{\fill} \\}
    
    \begin{Verbatim}[commandchars=\\\{\}]
{\color{incolor}In [{\color{incolor}8}]:} \PY{k}{def} \PY{n+nf}{iterative}\PY{p}{(}\PY{n}{img}\PY{p}{)}\PY{p}{:}
            
            \PY{n}{image} \PY{o}{=} \PY{n}{copy}\PY{o}{.}\PY{n}{deepcopy}\PY{p}{(}\PY{n}{img}\PY{p}{)}              \PY{c+c1}{\PYZsh{} create a copy of the image matrix}
            \PY{k}{for} \PY{n}{x} \PY{o+ow}{in} \PY{n+nb}{range}\PY{p}{(}\PY{n}{image}\PY{o}{.}\PY{n}{shape}\PY{p}{[}\PY{l+m+mi}{0}\PY{p}{]}\PY{p}{)}\PY{p}{:}
                \PY{k}{for} \PY{n}{y} \PY{o+ow}{in} \PY{n+nb}{range}\PY{p}{(}\PY{n}{image}\PY{o}{.}\PY{n}{shape}\PY{p}{[}\PY{l+m+mi}{1}\PY{p}{]}\PY{p}{)}\PY{p}{:}
                    \PY{k}{if} \PY{n}{x} \PY{o}{\PYZlt{}} \PY{n}{image}\PY{o}{.}\PY{n}{shape}\PY{p}{[}\PY{l+m+mi}{0}\PY{p}{]}\PY{o}{/}\PY{l+m+mi}{2} \PY{o+ow}{and} \PY{n}{y} \PY{o}{\PYZlt{}} \PY{n}{image}\PY{o}{.}\PY{n}{shape}\PY{p}{[}\PY{l+m+mi}{1}\PY{p}{]}\PY{o}{/}\PY{l+m+mi}{2}\PY{p}{:}
                        \PY{n}{image}\PY{p}{[}\PY{n}{x}\PY{p}{,}\PY{n}{y}\PY{p}{]} \PY{o}{=} \PY{n}{image}\PY{p}{[}\PY{n}{x}\PY{p}{,}\PY{n}{y}\PY{p}{]} \PY{o}{*} \PY{p}{[}\PY{l+m+mi}{0}\PY{p}{,}\PY{l+m+mi}{1}\PY{p}{,}\PY{l+m+mi}{1}\PY{p}{]}    \PY{c+c1}{\PYZsh{}removing the red channel}
                    \PY{k}{elif} \PY{n}{x} \PY{o}{\PYZgt{}} \PY{n}{image}\PY{o}{.}\PY{n}{shape}\PY{p}{[}\PY{l+m+mi}{0}\PY{p}{]}\PY{o}{/}\PY{l+m+mi}{2} \PY{o+ow}{and} \PY{n}{y} \PY{o}{\PYZlt{}} \PY{n}{image}\PY{o}{.}\PY{n}{shape}\PY{p}{[}\PY{l+m+mi}{1}\PY{p}{]}\PY{o}{/}\PY{l+m+mi}{2}\PY{p}{:}
                        \PY{n}{image}\PY{p}{[}\PY{n}{x}\PY{p}{,}\PY{n}{y}\PY{p}{]} \PY{o}{=} \PY{n}{image}\PY{p}{[}\PY{n}{x}\PY{p}{,}\PY{n}{y}\PY{p}{]} \PY{o}{*} \PY{p}{[}\PY{l+m+mi}{1}\PY{p}{,}\PY{l+m+mi}{0}\PY{p}{,}\PY{l+m+mi}{1}\PY{p}{]}    \PY{c+c1}{\PYZsh{}removing the green channel}
                    \PY{k}{elif} \PY{n}{x} \PY{o}{\PYZlt{}} \PY{n}{image}\PY{o}{.}\PY{n}{shape}\PY{p}{[}\PY{l+m+mi}{0}\PY{p}{]}\PY{o}{/}\PY{l+m+mi}{2} \PY{o+ow}{and} \PY{n}{y} \PY{o}{\PYZgt{}} \PY{n}{image}\PY{o}{.}\PY{n}{shape}\PY{p}{[}\PY{l+m+mi}{1}\PY{p}{]}\PY{o}{/}\PY{l+m+mi}{2}\PY{p}{:}
                        \PY{n}{image}\PY{p}{[}\PY{n}{x}\PY{p}{,}\PY{n}{y}\PY{p}{]} \PY{o}{=} \PY{n}{image}\PY{p}{[}\PY{n}{x}\PY{p}{,}\PY{n}{y}\PY{p}{]} \PY{o}{*} \PY{p}{[}\PY{l+m+mi}{1}\PY{p}{,}\PY{l+m+mi}{1}\PY{p}{,}\PY{l+m+mi}{0}\PY{p}{]}    \PY{c+c1}{\PYZsh{}removing the blue channel}
                    \PY{k}{else}\PY{p}{:}
                        \PY{k}{pass}
            \PY{k}{return} \PY{n}{image}
        
        \PY{k}{def} \PY{n+nf}{vectorized}\PY{p}{(}\PY{n}{img}\PY{p}{)}\PY{p}{:}
            
            \PY{n}{image} \PY{o}{=} \PY{n}{copy}\PY{o}{.}\PY{n}{deepcopy}\PY{p}{(}\PY{n}{img}\PY{p}{)}
            \PY{n}{a} \PY{o}{=} \PY{n+nb}{int}\PY{p}{(}\PY{n}{image}\PY{o}{.}\PY{n}{shape}\PY{p}{[}\PY{l+m+mi}{0}\PY{p}{]}\PY{o}{/}\PY{l+m+mi}{2}\PY{p}{)}
            \PY{n}{b} \PY{o}{=} \PY{n+nb}{int}\PY{p}{(}\PY{n}{image}\PY{o}{.}\PY{n}{shape}\PY{p}{[}\PY{l+m+mi}{1}\PY{p}{]}\PY{o}{/}\PY{l+m+mi}{2}\PY{p}{)}
            \PY{n}{image}\PY{p}{[}\PY{p}{:}\PY{n}{a}\PY{p}{,}\PY{p}{:}\PY{n}{b}\PY{p}{]} \PY{o}{=} \PY{n}{image}\PY{p}{[}\PY{p}{:}\PY{n}{a}\PY{p}{,}\PY{p}{:}\PY{n}{b}\PY{p}{]}\PY{o}{*}\PY{p}{[}\PY{l+m+mi}{0}\PY{p}{,}\PY{l+m+mi}{1}\PY{p}{,}\PY{l+m+mi}{1}\PY{p}{]}
            \PY{n}{image}\PY{p}{[}\PY{n}{a}\PY{p}{:}\PY{p}{,}\PY{p}{:}\PY{n}{b}\PY{p}{]} \PY{o}{=} \PY{n}{image}\PY{p}{[}\PY{n}{a}\PY{p}{:}\PY{p}{,}\PY{p}{:}\PY{n}{b}\PY{p}{]}\PY{o}{*}\PY{p}{[}\PY{l+m+mi}{1}\PY{p}{,}\PY{l+m+mi}{0}\PY{p}{,}\PY{l+m+mi}{1}\PY{p}{]}
            \PY{n}{image}\PY{p}{[}\PY{p}{:}\PY{n}{a}\PY{p}{,}\PY{n}{b}\PY{p}{:}\PY{p}{]} \PY{o}{=} \PY{n}{image}\PY{p}{[}\PY{p}{:}\PY{n}{a}\PY{p}{,}\PY{n}{b}\PY{p}{:}\PY{p}{]}\PY{o}{*}\PY{p}{[}\PY{l+m+mi}{1}\PY{p}{,}\PY{l+m+mi}{1}\PY{p}{,}\PY{l+m+mi}{0}\PY{p}{]}
            
            \PY{k}{return} \PY{n}{image}
\end{Verbatim}


    \begin{Verbatim}[commandchars=\\\{\}]
{\color{incolor}In [{\color{incolor}9}]:} \PY{l+s+sd}{\PYZdq{}\PYZdq{}\PYZdq{}}
        \PY{l+s+sd}{The code for this problem is posted on Piazza. }
        \PY{l+s+sd}{Sign up for the course if you have not. Then find}
        \PY{l+s+sd}{the function definition included in the post }
        \PY{l+s+sd}{\PYZsq{}Welcome to CSE252A\PYZsq{} to complete this problem.}
        \PY{l+s+sd}{This is the only cell you need to edit for this problem.}
        \PY{l+s+sd}{\PYZdq{}\PYZdq{}\PYZdq{}}
        \PY{k}{def} \PY{n+nf}{piazza}\PY{p}{(}\PY{p}{)}\PY{p}{:}
            \PY{n}{start} \PY{o}{=} \PY{n}{time}\PY{o}{.}\PY{n}{time}\PY{p}{(}\PY{p}{)}
            \PY{n}{image\PYZus{}iterative} \PY{o}{=} \PY{n}{iterative}\PY{p}{(}\PY{n}{img}\PY{p}{)} 
            \PY{n}{end} \PY{o}{=} \PY{n}{time}\PY{o}{.}\PY{n}{time}\PY{p}{(}\PY{p}{)}
            \PY{n+nb}{print}\PY{p}{(}\PY{l+s+s2}{\PYZdq{}}\PY{l+s+s2}{Iterative method took }\PY{l+s+si}{\PYZob{}0\PYZcb{}}\PY{l+s+s2}{ seconds}\PY{l+s+s2}{\PYZdq{}}\PY{o}{.}\PY{n}{format}\PY{p}{(}\PY{n}{end}\PY{o}{\PYZhy{}}\PY{n}{start}\PY{p}{)}\PY{p}{)}
            \PY{n}{start} \PY{o}{=} \PY{n}{time}\PY{o}{.}\PY{n}{time}\PY{p}{(}\PY{p}{)}
            \PY{n}{image\PYZus{}vectorized} \PY{o}{=} \PY{n}{vectorized}\PY{p}{(}\PY{n}{img}\PY{p}{)} 
            \PY{n}{end} \PY{o}{=} \PY{n}{time}\PY{o}{.}\PY{n}{time}\PY{p}{(}\PY{p}{)}
            \PY{n+nb}{print}\PY{p}{(}\PY{l+s+s2}{\PYZdq{}}\PY{l+s+s2}{Vectorized method took }\PY{l+s+si}{\PYZob{}0\PYZcb{}}\PY{l+s+s2}{ seconds}\PY{l+s+s2}{\PYZdq{}}\PY{o}{.}\PY{n}{format}\PY{p}{(}\PY{n}{end}\PY{o}{\PYZhy{}}\PY{n}{start}\PY{p}{)}\PY{p}{)}
            \PY{k}{return} \PY{n}{image\PYZus{}iterative}\PY{p}{,} \PY{n}{image\PYZus{}vectorized}
        
        
        \PY{c+c1}{\PYZsh{} Run the function}
        \PY{n}{image\PYZus{}iterative}\PY{p}{,} \PY{n}{image\PYZus{}vectorized} \PY{o}{=} \PY{n}{piazza}\PY{p}{(}\PY{p}{)}
\end{Verbatim}


    \begin{Verbatim}[commandchars=\\\{\}]
Iterative method took 0.7119317054748535 seconds
Vectorized method took 0.005984067916870117 seconds

    \end{Verbatim}

    \begin{Verbatim}[commandchars=\\\{\}]
{\color{incolor}In [{\color{incolor}10}]:} \PY{c+c1}{\PYZsh{} Plotting the results in sepearate subplots}
         
         \PY{n}{plt}\PY{o}{.}\PY{n}{subplot}\PY{p}{(}\PY{l+m+mi}{1}\PY{p}{,} \PY{l+m+mi}{3}\PY{p}{,} \PY{l+m+mi}{1}\PY{p}{)}  \PY{c+c1}{\PYZsh{} create (1x3) subplots, indexing from 1}
         \PY{n}{plt}\PY{o}{.}\PY{n}{imshow}\PY{p}{(}\PY{n}{img}\PY{p}{)}       \PY{c+c1}{\PYZsh{} original image}
         
         \PY{n}{plt}\PY{o}{.}\PY{n}{subplot}\PY{p}{(}\PY{l+m+mi}{1}\PY{p}{,} \PY{l+m+mi}{3}\PY{p}{,} \PY{l+m+mi}{2}\PY{p}{)}
         \PY{n}{plt}\PY{o}{.}\PY{n}{imshow}\PY{p}{(}\PY{n}{image\PYZus{}iterative}\PY{p}{)}
         
         \PY{n}{plt}\PY{o}{.}\PY{n}{subplot}\PY{p}{(}\PY{l+m+mi}{1}\PY{p}{,} \PY{l+m+mi}{3}\PY{p}{,} \PY{l+m+mi}{3}\PY{p}{)}
         \PY{n}{plt}\PY{o}{.}\PY{n}{imshow}\PY{p}{(}\PY{n}{image\PYZus{}vectorized}\PY{p}{)}
         
         \PY{n}{plt}\PY{o}{.}\PY{n}{show}\PY{p}{(}\PY{p}{)}           \PY{c+c1}{\PYZsh{}displays the subplots}
         
         \PY{n}{plt}\PY{o}{.}\PY{n}{imsave}\PY{p}{(}\PY{l+s+s2}{\PYZdq{}}\PY{l+s+s2}{multicolor\PYZus{}Lenna.png}\PY{l+s+s2}{\PYZdq{}}\PY{p}{,}\PY{n}{image\PYZus{}vectorized}\PY{p}{)}    \PY{c+c1}{\PYZsh{}Saving an image}
\end{Verbatim}


    \begin{center}
    \adjustimage{max size={0.9\linewidth}{0.9\paperheight}}{output_19_0.png}
    \end{center}
    { \hspace*{\fill} \\}
    
    \subsection{Problem 2 Further Image Manipulation
(5pts)}\label{problem-2-further-image-manipulation-5pts}

In this problem you will reuse the image "Lenna.png". Being a colored
image, this image has three channels, corresponding to the primary
colors of red, green and blue. Import this image and write your
implementation for extracting each of these channels separately to
create 2D images. This means that from the nxnx3 shaped image, you'll
get 3 matrices of the shape nxn (Note that it's two dimensional).

Now, write a function to merge all these images back into a colored 3D
image. The orginal image has a warm color tone, being more reddish. What
will the image look like if you exchange the reds with the blues? Merge
the 2D images first in original order of channels (RGB) and then with
red swapped with blue (BGR).

Finally, you will have \textbf{six images}, 1 original, 3 obtained from
channels, and 2 from merging. Using these 6 images, create one single
image by tiling them together \textbf{without using loops}. The image
will have 2x3 tiles making the shape of the final image
(2*512)x(3*512)x3. The order in which the images are tiled does not
matter. Display this image.

    \begin{Verbatim}[commandchars=\\\{\}]
{\color{incolor}In [{\color{incolor}11}]:} \PY{k+kn}{import} \PY{n+nn}{numpy} \PY{k}{as} \PY{n+nn}{np}
         \PY{k+kn}{import} \PY{n+nn}{matplotlib}\PY{n+nn}{.}\PY{n+nn}{pyplot} \PY{k}{as} \PY{n+nn}{plt}
         \PY{k+kn}{import} \PY{n+nn}{copy}
         \PY{n}{plt}\PY{o}{.}\PY{n}{rcParams}\PY{p}{[}\PY{l+s+s1}{\PYZsq{}}\PY{l+s+s1}{image.cmap}\PY{l+s+s1}{\PYZsq{}}\PY{p}{]} \PY{o}{=} \PY{l+s+s1}{\PYZsq{}}\PY{l+s+s1}{gray}\PY{l+s+s1}{\PYZsq{}} \PY{c+c1}{\PYZsh{} Necessary to override default matplot behaviour}
\end{Verbatim}


    \begin{Verbatim}[commandchars=\\\{\}]
{\color{incolor}In [{\color{incolor}12}]:} \PY{c+c1}{\PYZsh{} Write your code here. Import the image and define the required funtions.}
         
         \PY{n}{image} \PY{o}{=} \PY{k+kc}{None}
         \PY{c+c1}{\PYZsh{}Import image here}
         \PY{n}{image} \PY{o}{=} \PY{n}{plt}\PY{o}{.}\PY{n}{imread}\PY{p}{(}\PY{l+s+s1}{\PYZsq{}}\PY{l+s+s1}{Lenna.png}\PY{l+s+s1}{\PYZsq{}}\PY{p}{)}
         
         \PY{k}{def} \PY{n+nf}{getChannel}\PY{p}{(}\PY{n}{img}\PY{p}{,}\PY{n}{channel}\PY{p}{)}\PY{p}{:}
             \PY{l+s+sd}{\PYZsq{}\PYZsq{}\PYZsq{}}
         \PY{l+s+sd}{    Function for extracting 2D image corresponding to a }
         \PY{l+s+sd}{    channel number from a color image}
         \PY{l+s+sd}{    \PYZsq{}\PYZsq{}\PYZsq{}}
             \PY{n}{image} \PY{o}{=} \PY{n}{copy}\PY{o}{.}\PY{n}{deepcopy}\PY{p}{(}\PY{n}{img}\PY{p}{)} \PY{c+c1}{\PYZsh{}Create a copy so as to not change the original image}
             \PY{l+s+sd}{\PYZdq{}\PYZdq{}\PYZdq{}}
         \PY{l+s+sd}{    Picking all rows and cols of a specific channel only}
         \PY{l+s+sd}{    Note indexing starts with 0 in python, but the test code }
         \PY{l+s+sd}{    in this notebook uses indexing from 1, hence subtracting 1 from channel}
         \PY{l+s+sd}{    \PYZdq{}\PYZdq{}\PYZdq{}} 
             \PY{n}{img\PYZus{}ch} \PY{o}{=} \PY{n}{image}\PY{p}{[}\PY{p}{:}\PY{p}{,} \PY{p}{:}\PY{p}{,} \PY{n}{channel}\PY{o}{\PYZhy{}}\PY{l+m+mi}{1}\PY{p}{]}
             
             \PY{k}{return} \PY{n}{img\PYZus{}ch}
         
         
         \PY{k}{def} \PY{n+nf}{mergeChannels}\PY{p}{(}\PY{n}{image1}\PY{p}{,}\PY{n}{image2}\PY{p}{,}\PY{n}{image3}\PY{p}{)}\PY{p}{:}
             \PY{l+s+sd}{\PYZsq{}\PYZsq{}\PYZsq{}Function for merging three single channels images to form a color image\PYZsq{}\PYZsq{}\PYZsq{}}
             \PY{c+c1}{\PYZsh{} Expanding dimension of each 2\PYZhy{}D image}
             \PY{n}{image1} \PY{o}{=} \PY{n}{np}\PY{o}{.}\PY{n}{expand\PYZus{}dims}\PY{p}{(}\PY{n}{image1}\PY{p}{,} \PY{n}{axis}\PY{o}{=}\PY{l+m+mi}{2}\PY{p}{)}
             \PY{n}{image2} \PY{o}{=} \PY{n}{np}\PY{o}{.}\PY{n}{expand\PYZus{}dims}\PY{p}{(}\PY{n}{image2}\PY{p}{,} \PY{n}{axis}\PY{o}{=}\PY{l+m+mi}{2}\PY{p}{)}
             \PY{n}{image3} \PY{o}{=} \PY{n}{np}\PY{o}{.}\PY{n}{expand\PYZus{}dims}\PY{p}{(}\PY{n}{image3}\PY{p}{,} \PY{n}{axis}\PY{o}{=}\PY{l+m+mi}{2}\PY{p}{)}
             
             \PY{c+c1}{\PYZsh{} Merging using numpy.concatenate function along the channel dimension}
             \PY{n}{image\PYZus{}merged} \PY{o}{=} \PY{n}{np}\PY{o}{.}\PY{n}{concatenate}\PY{p}{(}\PY{p}{(}\PY{n}{image1}\PY{p}{,} \PY{n}{image2}\PY{p}{,} \PY{n}{image3}\PY{p}{)}\PY{p}{,} \PY{n}{axis}\PY{o}{=}\PY{l+m+mi}{2}\PY{p}{)}
             
             \PY{k}{return} \PY{n}{image\PYZus{}merged}
\end{Verbatim}


    \begin{Verbatim}[commandchars=\\\{\}]
{\color{incolor}In [{\color{incolor}13}]:} \PY{c+c1}{\PYZsh{} Test your function}
         
         \PY{c+c1}{\PYZsh{} getChannel returns a 2d image}
         \PY{k}{assert} \PY{n+nb}{len}\PY{p}{(}\PY{n}{getChannel}\PY{p}{(}\PY{n}{image}\PY{p}{,}\PY{l+m+mi}{1}\PY{p}{)}\PY{o}{.}\PY{n}{shape}\PY{p}{)} \PY{o}{==} \PY{l+m+mi}{2}   
         \PY{c+c1}{\PYZsh{} mergeChannels returns a 3d image}
         \PY{k}{assert} \PY{n+nb}{len}\PY{p}{(}\PY{n}{mergeChannels}\PY{p}{(}
                 \PY{n}{getChannel}\PY{p}{(}\PY{n}{image}\PY{p}{,}\PY{l+m+mi}{1}\PY{p}{)}\PY{p}{,}\PY{n}{getChannel}\PY{p}{(}\PY{n}{image}\PY{p}{,}\PY{l+m+mi}{2}\PY{p}{)}\PY{p}{,}\PY{n}{getChannel}\PY{p}{(}\PY{n}{image}\PY{p}{,}\PY{l+m+mi}{3}\PY{p}{)}\PY{p}{)}\PY{o}{.}\PY{n}{shape}\PY{p}{)} \PY{o}{==} \PY{l+m+mi}{3}
\end{Verbatim}


    \begin{Verbatim}[commandchars=\\\{\}]
{\color{incolor}In [{\color{incolor}14}]:} \PY{l+s+sd}{\PYZdq{}\PYZdq{}\PYZdq{}}
         \PY{l+s+sd}{Write your code here for tiling the six images to make a single image }
         \PY{l+s+sd}{and displaying it. Notice that the images returned by getChannel will be }
         \PY{l+s+sd}{2 dimensional. To tile them together with RGB images, you might need to }
         \PY{l+s+sd}{change it to a 3 dimensional image. This can be done using np.expand\PYZus{}dims }
         \PY{l+s+sd}{and specifying the axis as an argument.}
         \PY{l+s+sd}{\PYZdq{}\PYZdq{}\PYZdq{}}
         
         
         \PY{c+c1}{\PYZsh{} image is the original unaltered RGB image}
         \PY{n}{image\PYZus{}r} \PY{o}{=} \PY{n}{getChannel}\PY{p}{(}\PY{n}{image}\PY{p}{,} \PY{l+m+mi}{1}\PY{p}{)}
         \PY{n}{image\PYZus{}g} \PY{o}{=} \PY{n}{getChannel}\PY{p}{(}\PY{n}{image}\PY{p}{,} \PY{l+m+mi}{2}\PY{p}{)}
         \PY{n}{image\PYZus{}b} \PY{o}{=} \PY{n}{getChannel}\PY{p}{(}\PY{n}{image}\PY{p}{,} \PY{l+m+mi}{3}\PY{p}{)}
         
         \PY{c+c1}{\PYZsh{} converting back to 3\PYZhy{}D original in RGB format}
         \PY{n}{image\PYZus{}rgb} \PY{o}{=} \PY{n}{mergeChannels}\PY{p}{(}\PY{n}{image\PYZus{}r}\PY{p}{,} \PY{n}{image\PYZus{}g}\PY{p}{,} \PY{n}{image\PYZus{}b}\PY{p}{)}
         
         \PY{c+c1}{\PYZsh{} converting back to 3\PYZhy{}D original in BGR format}
         \PY{n}{image\PYZus{}bgr} \PY{o}{=} \PY{n}{mergeChannels}\PY{p}{(}\PY{n}{image\PYZus{}b}\PY{p}{,} \PY{n}{image\PYZus{}g}\PY{p}{,} \PY{n}{image\PYZus{}r}\PY{p}{)}
         
         \PY{c+c1}{\PYZsh{} converting 2\PYZhy{}D image splits into a 3\PYZhy{}D shape for tiling purpose}
         \PY{n}{image\PYZus{}r\PYZus{}3ch} \PY{o}{=} \PY{n}{np}\PY{o}{.}\PY{n}{zeros}\PY{p}{(}\PY{n}{image}\PY{o}{.}\PY{n}{shape}\PY{p}{)}
         \PY{n}{image\PYZus{}g\PYZus{}3ch} \PY{o}{=} \PY{n}{np}\PY{o}{.}\PY{n}{zeros}\PY{p}{(}\PY{n}{image}\PY{o}{.}\PY{n}{shape}\PY{p}{)}
         \PY{n}{image\PYZus{}b\PYZus{}3ch} \PY{o}{=} \PY{n}{np}\PY{o}{.}\PY{n}{zeros}\PY{p}{(}\PY{n}{image}\PY{o}{.}\PY{n}{shape}\PY{p}{)}
         \PY{n}{image\PYZus{}r\PYZus{}3ch}\PY{p}{[}\PY{p}{:}\PY{p}{,} \PY{p}{:}\PY{p}{,} \PY{l+m+mi}{0}\PY{p}{]} \PY{o}{=} \PY{n}{image\PYZus{}r}
         \PY{n}{image\PYZus{}g\PYZus{}3ch}\PY{p}{[}\PY{p}{:}\PY{p}{,} \PY{p}{:}\PY{p}{,} \PY{l+m+mi}{1}\PY{p}{]} \PY{o}{=} \PY{n}{image\PYZus{}g}
         \PY{n}{image\PYZus{}b\PYZus{}3ch}\PY{p}{[}\PY{p}{:}\PY{p}{,} \PY{p}{:}\PY{p}{,} \PY{l+m+mi}{2}\PY{p}{]} \PY{o}{=} \PY{n}{image\PYZus{}b}
         
         
         \PY{c+c1}{\PYZsh{} tiling all the images}
         \PY{n}{tile\PYZus{}row\PYZus{}1} \PY{o}{=} \PY{n}{np}\PY{o}{.}\PY{n}{concatenate}\PY{p}{(}\PY{p}{(}\PY{n}{image\PYZus{}r\PYZus{}3ch}\PY{p}{,} \PY{n}{image\PYZus{}g\PYZus{}3ch}\PY{p}{,} \PY{n}{image\PYZus{}b\PYZus{}3ch}\PY{p}{)}\PY{p}{,} \PY{n}{axis}\PY{o}{=}\PY{l+m+mi}{1}\PY{p}{)}
         \PY{n}{tile\PYZus{}row\PYZus{}2} \PY{o}{=} \PY{n}{np}\PY{o}{.}\PY{n}{concatenate}\PY{p}{(}\PY{p}{(}\PY{n}{image}\PY{p}{,} \PY{n}{image\PYZus{}rgb}\PY{p}{,} \PY{n}{image\PYZus{}bgr}\PY{p}{)}\PY{p}{,} \PY{n}{axis}\PY{o}{=}\PY{l+m+mi}{1}\PY{p}{)}
         \PY{n}{tile\PYZus{}image} \PY{o}{=} \PY{n}{np}\PY{o}{.}\PY{n}{concatenate}\PY{p}{(}\PY{p}{(}\PY{n}{tile\PYZus{}row\PYZus{}1}\PY{p}{,} \PY{n}{tile\PYZus{}row\PYZus{}2}\PY{p}{)}\PY{p}{,} \PY{n}{axis}\PY{o}{=}\PY{l+m+mi}{0}\PY{p}{)}
         
         \PY{c+c1}{\PYZsh{} plot}
         \PY{n+nb}{print}\PY{p}{(}\PY{l+s+s2}{\PYZdq{}}\PY{l+s+s2}{Shape of tile image: }\PY{l+s+si}{\PYZob{}\PYZcb{}}\PY{l+s+s2}{\PYZdq{}}\PY{o}{.}\PY{n}{format}\PY{p}{(}\PY{n}{tile\PYZus{}image}\PY{o}{.}\PY{n}{shape}\PY{p}{)}\PY{p}{)}
         \PY{n}{plt}\PY{o}{.}\PY{n}{title}\PY{p}{(}\PY{l+s+s2}{\PYZdq{}}\PY{l+s+s2}{Row1: R, G, B; Row2: original, mergedRGB, mergedBGR}\PY{l+s+s2}{\PYZdq{}}\PY{p}{)}
         \PY{n}{plt}\PY{o}{.}\PY{n}{imshow}\PY{p}{(}\PY{n}{tile\PYZus{}image}\PY{p}{)}
         \PY{n}{plt}\PY{o}{.}\PY{n}{imsave}\PY{p}{(}\PY{l+s+s2}{\PYZdq{}}\PY{l+s+s2}{tile\PYZus{}Lenna.png}\PY{l+s+s2}{\PYZdq{}}\PY{p}{,} \PY{n}{tile\PYZus{}image}\PY{p}{)}
\end{Verbatim}


    \begin{Verbatim}[commandchars=\\\{\}]
Shape of tile image: (1024, 1536, 3)

    \end{Verbatim}

    \begin{center}
    \adjustimage{max size={0.9\linewidth}{0.9\paperheight}}{output_24_1.png}
    \end{center}
    { \hspace*{\fill} \\}
    
    \begin{center}\rule{0.5\linewidth}{\linethickness}\end{center}

** Submission Instructions**\\
Remember to submit a pdf version of this notebook to Gradescope. You can
find the export option at File \(\rightarrow\) Download as
\(\rightarrow\) PDF via LaTeX


    % Add a bibliography block to the postdoc
    
    
    
    \end{document}
